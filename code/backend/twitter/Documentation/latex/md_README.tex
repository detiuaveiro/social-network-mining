\subsection*{Tweet object}


\begin{DoxyItemize}
\item On backend/twitter/examples/tweet\+\_\+object.\+json
\item How to add to mongo\+: 
\begin{DoxyCode}
> use twitter
> db.createCollection("tweets")
> db.tweets.insert([<content on backend/twitter/examples/tweet\_object.json>])
\end{DoxyCode}

\end{DoxyItemize}

\subsection*{User object}


\begin{DoxyItemize}
\item On backend/twitter/examples/user\+\_\+object.\+json
\item How to add to mongo\+: 
\begin{DoxyCode}
> use twitter
> db.createCollection("users")
> db.users.insert([<content on backend/twitter/examples/users\_object.json>])
\end{DoxyCode}

\end{DoxyItemize}

\subsection*{A\+T\+E\+N\+T\+I\+O\+N!}


\begin{DoxyItemize}
\item When we are inserting a new tweet object, it is necessary that all arguments defined on the django models are placed (even if their value is set to null), because of the db integrety
\item The twitter id\textquotesingle{}s with wish we are working on the rest api correspond to the str\+\_\+id of the tweets objects
\end{DoxyItemize}

\section*{Django Multi\+D\+BS and D\+Bs configuration}

\subsection*{Useful Link}


\begin{DoxyItemize}
\item \href{https://docs.djangoproject.com/en/3.0/topics/db/multi-db/}{\tt https\+://docs.\+djangoproject.\+com/en/3.\+0/topics/db/multi-\/db/}
\end{DoxyItemize}

\subsection*{Makemigrations and migrate operations}


\begin{DoxyItemize}
\item Now its necessary define which db you want to do migrate operation, because default DB it is not defined
\item DB Names are defined on D\+A\+T\+A\+B\+A\+S\+ES dictionary (keys) on settings.\+py
\end{DoxyItemize}


\begin{DoxyCode}
$ rm -rf api/migrations
$ python3 manage.py makemigrations api
$ python3 manage.py migrate --database postgres <or>  python3 manage.py migrate --database mongo 
\end{DoxyCode}


\subsection*{Mongo}


\begin{DoxyItemize}
\item Create user (for development)
\end{DoxyItemize}
\begin{DoxyEnumerate}
\item Create a user admin (to manage other users) 
\begin{DoxyCode}
$ mongo
$ use admin;
$     db.createUser(
      \{
        user: "admin",
        pwd: "admin",
        roles: [ \{ role: "root", db: "admin" \} ]
      \}
    )
\end{DoxyCode}

\item Using admin user, create another user (django user) 
\begin{DoxyCode}
$ mongo --port 27017 -u "admin" -p "admin" --authenticationDatabase "admin"
$ use twitter;
$     db.createUser(
      \{
        user: "admin",
        pwd: "admin",
        roles: [ \{ role: "dbOwner" , db: "twitter" \} ]
      \}
    )
\end{DoxyCode}

\end{DoxyEnumerate}
\begin{DoxyItemize}
\item Access DB (after user creation) 
\begin{DoxyCode}
mongo --port 27017 -u "admin" -p "admin" --authenticationDatabase "twitter"
\end{DoxyCode}

\end{DoxyItemize}

\subsection*{Postgres}


\begin{DoxyItemize}
\item Create user (for development) 
\begin{DoxyCode}
$ sudo su postgres -c psql
# create database postgres;
# create user admin with password 'admin';
# GRANT ALL PRIVILEGES ON DATABASE postgres TO admin;
# \(\backslash\)c postgres
# GRANT ALL ON ALL TABLES IN SCHEMA public to admin;
# GRANT ALL ON ALL SEQUENCES IN SCHEMA public to admin;
# GRANT ALL ON ALL FUNCTIONS IN SCHEMA public to admin;
\end{DoxyCode}

\item Access DB (after user creation) 
\begin{DoxyCode}
$ psql -d twitter\_postgres -U admin -W 
\end{DoxyCode}

\end{DoxyItemize}

\section*{Policies Object}


\begin{DoxyItemize}
\item Items\+:
\begin{DoxyItemize}
\item id \+: int
\item A\+P\+I\+\_\+type \+: str (T\+W\+I\+T\+T\+ER OR I\+N\+S\+T\+A\+G\+R\+AM)
\item filter \+: str (U\+S\+E\+R\+N\+A\+ME OR K\+E\+Y\+W\+O\+R\+DS)
\item name \+: str
\item tags \+: str\mbox{[}\mbox{]}
\item bots \+: int\mbox{[}\mbox{]}
\end{DoxyItemize}
\item Example in dictionary 
\begin{DoxyCode}
\{ "id" : 1, "API\_type": "Twitter", "filter": "Target", "name": "Politica", "tags": ["PSD", "CDS"], "bots":
       [1, 2] \}
\end{DoxyCode}
 
\end{DoxyItemize}